%%%%%%%%%%%%%%%%%%%%%%%%%%%%%%%%%%%%%%%%%
% Short Sectioned Assignment
% LaTeX Template
% Version 1.0 (5/5/12)
%
% This template has been downloaded from:
% http://www.LaTeXTemplates.com
%
% Original author:
% Frits Wenneker (http://www.howtotex.com)
%
% License:
% CC BY-NC-SA 3.0 (http://creativecommons.org/licenses/by-nc-sa/3.0/)
%
%%%%%%%%%%%%%%%%%%%%%%%%%%%%%%%%%%%%%%%%%

%----------------------------------------------------------------------------------------
%	PACKAGES AND OTHER DOCUMENT CONFIGURATIONS
%----------------------------------------------------------------------------------------

\documentclass[paper=a4, fontsize=10pt]{scrartcl} % A4 paper and 11pt font size
\usepackage{geometry}
\geometry{margin=2cm}
\usepackage[T1]{fontenc} % Use 8-bit encoding that has 256 glyphs
%\usepackage{fourier} % Use the Adobe Utopia font for the document - comment this line to return to the LaTeX default
\usepackage[english]{babel} % English language/hyphenation
\usepackage[utf8]{inputenc} 
\usepackage{amsmath,amsfonts,amsthm} % Math packages

\usepackage{helvet}
\renewcommand{\familydefault}{\sfdefault}

\usepackage{lipsum} % Used for inserting dummy 'Lorem ipsum' text into the template
\usepackage{graphicx}

\usepackage{sectsty} % Allows customizing section commands
%\allsectionsfont{\centering \normalfont\scshape} % Make all sections centered, the default font and small caps
\geometry{left=1cm}
\usepackage{fancyhdr} % Custom headers and footers
\pagestyle{fancyplain} % Makes all pages in the document conform to the custom headers and footers
\fancyhead{} % No page header - if you want one, create it in the same way as the footers below
\fancyfoot[L]{} % Empty left footer
\fancyfoot[C]{} % Empty center footer
\fancyfoot[R]{\thepage} % Page numbering for right footer
\renewcommand{\headrulewidth}{0pt} % Remove header underlines
\renewcommand{\footrulewidth}{0pt} % Remove footer underlines
\setlength{\headheight}{13.6pt} % Customize the height of the header

\numberwithin{equation}{section} % Number equations within sections (i.e. 1.1, 1.2, 2.1, 2.2 instead of 1, 2, 3, 4)
\numberwithin{figure}{section} % Number figures within sections (i.e. 1.1, 1.2, 2.1, 2.2 instead of 1, 2, 3, 4)
\numberwithin{table}{section} % Number tables within sections (i.e. 1.1, 1.2, 2.1, 2.2 instead of 1, 2, 3, 4)

\setlength\parindent{0pt} % Removes all indentation from paragraphs - comment this line for an assignment with lots of text

%----------------------------------------------------------------------------------------
%	TITLE SECTION
%----------------------------------------------------------------------------------------

\newcommand{\horrule}[1]{\rule{\linewidth}{#1}} % Create horizontal rule command with 1 argument of height

\title{	
\normalfont \normalsize 
\textsc{University of Freiburg} \\ [25pt] % Your university, school and/or department name(s)
\horrule{0.5pt} \\[0.4cm] % Thin top horizontal rule
\huge Exercise 3: Mc \& TD \\ % The assignment title
\horrule{2pt} \\[0.5cm] % Thick bottom horizontal rule
}

\author{Nico Ott 4214197\\ Lior Fuks 4251285 \\Hendrik Vloet 4324249}

\date{\normalsize\today} % Today's date or a custom date

\begin{document}

\maketitle % Print the title

%----------------------------------------------------------------------------------------
%	PROBLEM 1
%----------------------------------------------------------------------------------------
\section{Task 1}
\subsection{a)}
Episode 1:
\begin{flalign*}
	V_1(2,1) = 5\\
	V_1(2,3) = 6\\
	V_1(1,3) = 7\\
	V_1(1,3) = 8\\
	V_1(1,4) = 9\\
	V_1(2,5) = 10\\
	\text{other states: 0}
\end{flalign*}
Episode 2:
\begin{flalign*}
	V_2(2,1) = 5+ \frac{1}{2}*(-102-5) = -48.5\\
	V_2(2,2) = 6 + \frac{1}{2}*(-101-6) = -47.5\\
	V_2(2,3) = 7 + \frac{1}{2}*(-100-7) = -46.5\\
	\text{other state: unchanged}
\end{flalign*}
Episode 3:
\begin{align*}
	V_3(2,1) = -48.5+\frac{1}{3}*(-5-(-48.5))= -30.7\\
	V_3(2,2) = -47.5+\frac{1}{3}*(-6-(-47.5)) = -29.7\\
	V_3(2,3) = -46.5+\frac{1}{3}*(-7-(-46.5)) = -28.7\\
	V_3(2,4) = 0+ \frac{1}{3}*(8-0) = 2.7\\
	V_3(1,4) = 9+ \frac{1}{3}*(9-9) = 9\\
	V_3(2,4) \leftarrow \text{no update, was visited before}\\
	\text{other state: unchanged}	
\end{align*}


\begin{center}

\textbf{Values after 3 Episodes of first visit MC}\\
\begin{tabular}{|c|c|c|c|c|}
\hline 
0 & 0 & 8 & 9 & 10 \\ 
\hline 
-30.7 & -29.7 & -28.7 & 2.7 & 0 \\ 
\hline 
0 & 0 & 0 & 0 & 0 \\ 
\hline 
\end{tabular} 

\end{center}
\subsection{b)}
TD-Error
\begin{align*}
	\delta _4 (2,1) = -1 + 0 - (30.7) = -29.7 \\
	\delta _4 (1,1) = -1 + 0 - (0) = -1 \\
	\delta _4 (1,2) = -1 + 8 - (0) = 7 \\
	\delta _4 (1,3) = -1 + 9 - (8) = 0 \\
	\delta _4 (1,4) = -1 + 10 - (9) =0  \\
	\delta _4 (1,5) = 10 + 0 -(10) = 0 \\
\end{align*}


%\section{Experiences}
%\begin{itemize}
%	\item \textbf{Nico}
%	\begin{itemize}
%		\item Invested Time:
%		\begin{itemize}
%			
%		\end{itemize}		
%		\item General:
%		
%		\end{itemize}
%	\end{itemize}
%
%	\item \textbf{Hendrik}
%	\begin{itemize}
%		\item Invested time:
%		\begin{itemize}
%			\item Meeting: 2h
%			\item Lecture 3: 3h
%			\item Exercise : $\sim$ 12h
%		\end{itemize}
%		\item General:
%		\begin{itemize}
%			\item understanding issues with the unit test structure of the python environment
%			\item I had trouble comprehending why we had to go for a deterministic policy, thanks to rr114\_uni-freiburg  for helping out!
%			\item Sadly, protocol morale for the google doc seems rather low. I added some things regarding the video lecture and the according discussion last time.		
%		\end{itemize}
%	\end{itemize}
%\end{itemize}
%	


\end{document}